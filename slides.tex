% use pdflatex -shell-escape slides.tex to compile
\documentclass[10pt]{beamer}
\usetheme{CambridgeUS}
%\usetheme[
%%% options passed to the outer theme
%    shownavsym          % show the navigation symbols
%]{AAUsimple}

\setbeamercolor{block title}{fg=gray!45!white,bg=darkred!80!black}
\overfullrule=5pt
% If you want to change the colors of the various elements in the theme, edit and uncomment the following lines
% Change the bar and sidebar colors:
%\setbeamercolor{AAUsimple}{fg=red!20,bg=red}
%\setbeamercolor{sidebar}{bg=red!20}
% Change the color of the structural elements:
%\setbeamercolor{structure}{fg=red}
% Change the frame title text color:
%\setbeamercolor{frametitle}{fg=blue}
% Change the normal text color background:
%\setbeamercolor{normal text}{fg=black,bg=gray!10}
% ... and you can of course change a lot more - see the beamer user
% manual.
\usepackage{animate}
\usepackage{tabularx}
\usepackage{epstopdf}
\usepackage{movie15}
\usepackage[utf8]{inputenc}
\usepackage[english]{babel}
\usepackage[T1]{fontenc}
% Or whatever. Note that the encoding and the font should match. If T1
% does not look nice, try deleting the line with the fontenc.
\usepackage{helvet}
\usepackage[helvet]{sfmath}
\usepackage{graphicx}
% For math font script
\usepackage[mathscr]{euscript}
\usepackage{algorithm,algorithmic}
\usepackage{color, xcolor}
\usepackage{tikz}
\usepackage{pgf}

% tikz setup for lattice graph
\usetikzlibrary{positioning,chains,fit,shapes,calc,snakes}
\usetikzlibrary{arrows,shadows,trees,automata}
% tikz animation
\usetikzlibrary{decorations.pathmorphing,through,backgrounds,petri}
\tikzset{
  basic/.style  = {draw, text width=2cm, drop shadow, font=\sffamily, rectangle},
  root/.style   = {basic, rounded corners=2pt, thin, align=center,
                   fill=red!30},
  level 2/.style = {basic, rounded corners=6pt, thin,align=center, fill=orange!30,
                   text width=8em},
  level 3/.style = {basic, thin, align=left, fill=pink!60, text width=8em},
  post/.style={->,>=stealth', thick},
  arc/.style={->,>=stealth' },
  node/.style={circle,draw,font=\sffamily\small},
  pile/.style={thin, ->, >=stealth', shorten <=2pt, shorten >=2pt},
  edge/.style={->, >=stealth', thin},
  edges/.style={->, shorten <=2pt, shorten >=2pt}
}
\input{defs.tex}

\setbeamercolor{title}{fg=white, bg=scred}

\title[Ph.D. Defense]{Constrained Geometric Approximation for Robot Planning and  Decentralized Formation Algorithm for Multi-Robot Systems}

\author[Yang Song]{
  \underline{Yang Song}\\
  Advisor: Jason M. O'Kane
}

\institute[
%  {\includegraphics[scale=0.2]{aau_segl}}\\ %insert a company, department or
%  university logo
USC
% Dept.\ of Computer Science and Engineering\\
  % University of South Carolina
] % optional - is placed in the bottom of the sidebar on every slide
{ % is placed on the bottom of the title page
  Dept. of Computer Science and Engineering\\
  University of South Carolina
  
  %there must be an empty line above this line - otherwise some unwanted space is added between the university and the country (I do not know why;( )
}

% specify a logo on the titlepage 
% institute command below
\pgfdeclareimage[height=1.5cm]{titlepagelogo}{sc_logo.pdf}
%\pgfdeclareimage[height=1.5cm]{titlepagelogo2}
\titlegraphic{% is placed on the bottom of the title page
  \pgfuseimage{titlepagelogo}
}

\begin{document}
% the titlepage
\begin{frame}
  \titlepage
\end{frame}
%%%%%%%%%%%%%%%%
\begin{frame}{Outlines}{}
\tableofcontents
\end{frame}
%%%%%%%%%%%%%%%%

\section{CGA for Robot Planning [ICRA 2012]}
\subsection[Overview]{Overview}
\begin{frame}{Constrained Geometric Approximation}
\begin{columns}
  \begin{column}{.65\textwidth}
    \begin{itemize}
    \item \textbf{Goal}:\\
    For an extremely simple robot with:
    \begin{itemize}
    \item computation limitations
    \item moving and sensing uncertainties
    \end{itemize}
    represent and reason about uncertainty in its own states efficiently.\\
    \item \textbf{Basic Idea}:\\
    Explicitly represent what the robot knows as an information state (\textit{I-state}).
    \item \textbf{Intuition}:\\
    Accelerate time-consuming operations by maintaining only an \textcolor[rgb]{1.00,0.00,0.00}{over-approximation} of the true
    I-state, and constraining this approximation
    to have a simple geometric form.\\
    \end{itemize}
  \end{column}
  \begin{column}{.35\textwidth}
    \begin{figure}
    \includegraphics[scale=0.2]{figs/srvq.jpg}
    \end{figure}
    SRV-1 Surveyor Robot
  \end{column}
\end{columns}
\end{frame}

\begin{frame}{Prior Work}
Related work
\begin{itemize}
\item Probabilistic representations: J. van den Berg,
  P. Abbeel, and K. Goldberg (LQG-MP, IJRR 2011)
\item Minimal Representations: B. Tovar, L. Guilamo, S. M. LaValle (GNT, WAFR 2004)
\item Geometric over-approximation: W. Xu, J. M. O'Kane (Wireless Netw 2012)
\end{itemize}
Contributions
\begin{itemize}
\item Define the range space $\mathcal{R}$ with formulation of the operations
\item Design algorithms for double-rectangle range space $\mathcal{R}_{drect}$
\item Conduct experiments to evaluate different range spaces
\end{itemize}
\end{frame}

\begin{frame}{Robot Model}
Assume that the robot could not observe its real state directly.  
It make its decisions by maintaining an \emph{I-state} $\eta_k$.
\begin{itemize}
\item State space: $X = \mathbb{R}^2$,  $X_{obt} \cup X_{free} \subseteq X$
\item Action space: $U = \mathbb{R}^2$, apply a bounded noise $\theta_k$ to the
  action $u_k \in U$
\item Set valued state transition function: $F(x, u)$
  \begin{equation}
    \label{eq:state-trans}
    F(x_k, u_k) = \left\{
      x_k + u_k + \theta_k
      \mid
      \theta_k \in \Theta(u_k)
    \right\} \cap X_{free}
  \end{equation}
  \begin{columns}
    \begin{column}{0.4\textwidth}
      \begin{figure}
        \centering
        \includegraphics[scale=0.2]{figs/istate.jpg}
        \caption{\scriptsize{The transition of robot's state $x_k$ with
            uncertainty [\emph{Planning Algorithms}, S. LaValle, 2006].}}
      \end{figure}
    \end{column}
    \begin{column}{0.5\textwidth}
      \begin{figure}
        \centering
        \includegraphics[scale=0.55]{figs/noisemodel}
        \caption{\scriptsize{The noise model $\Theta(u_k)$ involves bounded
            angular error $\delta_{ang}$ and bounded translational error
            $\delta_{trans}||u||$.}}
        \label{fig:noiseModel}
      \end{figure}
    \end{column}
  \end{columns}
\end{itemize}
\end{frame}


\begin{frame}{Robot Model (Cont'd)}
  \begin{itemize}
  \item Observation space: $Y = \mathbb{R}^2$
  \item Observation function: $h : X \to \pow(Y) $
  \item Observation Preimage: the set of states from which a given observation
    $y_k \in Y$ can be obtained
        \begin{equation}
          H(y_k) = \left\{ x_k \in X \mid y_k \in h(x_k), y_k \in Y \right\}
        \end{equation}
         \begin{figure}
      \centering
      \begin{tikzpicture}[scale=0.45]
        \draw[blue!20, fill=blue!20] (3, 3) circle (3);
        \draw[fill=violet] (3,3) circle (0.2);
        \draw[fill=red] (3,4.5) circle (0.1);
        \draw[fill=red] (3.2,1.7) circle (0.1);
        \draw[fill=red] (1,3) circle (0.1);
        \draw[fill=red] (2,0.5) circle (0.1);
        \draw[fill=red] (4,3.3) circle (0.1);
        \draw[fill=red] (4.4,4.8) circle (0.1);
      \end{tikzpicture}
      \caption{\scriptsize{The observation preimage set $H(y)$ (circle). Any states
          in this region (red points) can get observation of the landmark $y$
          (violet point).}}
    \end{figure}
  \end{itemize}
\end{frame}

\subsection[I-Space]{Information Space}
\begin{frame}{Information Space}
  \begin{definition}
  \small{An information state (I-state) $\eta$ is a set containing all possible
    states consistent with robot's sensor and action history.}
  \end{definition}
  \begin{definition}
  \small{The \emph{information space} (I-space) $\Ispace$ is the
  powerset of $X$, which contains all possible I-states.}
  \end{definition}
  \begin{columns}
    \begin{column}{0.4\textwidth}
       1. Action update function:
       $$X_{k+1}(\eta_{k}, u_k) =  \bigcup_{x_k \in \eta_k} F(x_k, u_k)$$
     \end{column}
    \begin{column}{0.5\textwidth}
      \input{tikzfigs/istate.tex}
    \end{column}
  \end{columns}
\end{frame}

\begin{frame}{Information Space}
  \begin{definition}
    \small{An information state (I-state) $\eta$ is a set containing all possible
      states consistent with robot's sensor and action history.}
  \end{definition}
  \begin{definition}
    \small{The \emph{information space} (I-space) $\Ispace$ is the
      powerset of $X$, which contains all possible I-states.}
  \end{definition}
  \begin{columns}
    \begin{column}{0.4\textwidth}
      2. Observation update function:
      $$\eta_{k+1} =   X_{k+1}(\eta_{k}, u_k) \cap H(y_{k})$$
    \end{column}
    \begin{column}{0.5\textwidth}
      \input{tikzfigs/istateobsv.tex}
    \end{column}
  \end{columns}  
 \end{frame}

\begin{frame}{Simulation: Navigation Task}
    \begin{block}{Ultimate Objective} 
       Form a plan $\pi$ for the robot to execute using the information available in
       the I-space.
    \end{block}

    \textbf{In a navigation task, we assume:}
    \begin{itemize}
    \item A point robot follows predefined waypoints guided by centroid point of the
      (approximated) I-state
    \item Robot can detect presence but not distance to the landmarks
    \item Landmarks are pseudo-randomly generated
    \item Initial (approximated) I-state is given
    \end{itemize}
\end{frame}

\begin{frame}{Example}{Navigation Task using I-state}
  \begin{center}
    \includemovie[toolbar,poster,autoplay]{12cm}{6cm}{videos/clutter_istate.mp4}
  \end{center}
\end{frame}  % video

%%%%%%%%%%%%%%%%%%%%%%%%%%%%%%%%%%%%%%%%%%%%%%%%%%%%%%%%%%%%%%%%%%%%%%%%%%%%%
\subsection[Range Space]{Contribution: Range Space}
\begin{frame}{Contribution: Range Space}
    \begin{definition}{\textbf{A range space}}
    $\mathcal{R} \subseteq \mathcal{I}$ is a set of I-states, equipped with two
    operations:
    \end{definition}
    \begin{columns}
        \begin{column}{.5\textwidth}
          \begin{enumerate}
          \item \emph{Approximate action update function} $T: \mathcal{R} \times U \to
            \mathcal{R}$:
            $$\bigcup_{x_k \in \eta_k} F(x_k, u_k) \subseteq T(A(\eta_k), u_k)$$
          \item \emph{Approximate observation update function} $O: \mathcal{R} \times
            Y \to \mathcal{R}$:
            $$\eta_k \cap H(y_k) \subseteq O(A(\eta_k), u_k)$$
          \end{enumerate}
          \textcolor{scred}{Intuition: always over-approximate I-state:
            $\eta_k \subseteq A(\eta_k)$}
        \end{column}
        \begin{column}{.4\textwidth}
            \begin{figure}
    \begin{tikzpicture}[scale=0.75]
      \begin{scope}
        \clip (0,0) rectangle (4,4);
        \draw[fill=green!20] (0,0) circle (3cm);
        \draw[fill=white] (0,0) circle (1cm);  
      \end{scope}
      \draw[blue] (0,0) -- (3,0) -- (3,3) -- (0,3) -- (0,0);
      \draw[red]  (1.5,1.5) circle (2.13cm);
      \node at (1.3,1.3) {$\eta_k$};    
    \end{tikzpicture} 
    \caption{\scriptsize{An \emph{I-state} $\eta_k$ (shaded region),
        over-approximations $A(\eta_k)$ using disk (red circle) and rectangle
        (blue). }}
\end{figure}
        \end{column}
    \end{columns}
\end{frame}

\begin{frame}{Example}{Rectangle Approximation of I-state}
  \begin{center}
    \includemovie[toolbar,poster,autoplay]{12cm}{6cm}{videos/clutter_rect.mp4}
  \end{center}
\end{frame}



\begin{frame}{Limitation}{Rectangle Approximation of I-state}
  \textbf{Problem: approximation accuracy}
   \begin{columns}
    \begin{column}{0.45\textwidth}
      \begin{figure}
        \includegraphics[width=0.5\linewidth]{figs/rect_approx}
      \end{figure}
    \end{column}
    \begin{column}{0.45\textwidth}
      \begin{figure}
        \includegraphics[width=0.5\linewidth]{figs/dbrect_approx}
      \end{figure}
    \end{column}
  \end{columns}
\end{frame}

\subsubsection[Double-Rectangle Range Space]{Double-Rectangle Range Space}

\begin{frame}{Double-Rectangle approximated I-state}
  To improve the approximation quality for non-convex \emph{I-states}, we
  proposed:
  \begin{itemize}
  \item a novel range space of \textcolor[rgb]{1.00,0.00,0.00}{double-rectangle}
    \begin{equation}
      \mathcal{R}_{drect} = \{ R_1 \cup R_2 \mid R_1, R_2 \in \mathcal{R}_{rect} \}
    \end{equation}
  \item an algorithm to overapproximate a polygon using double-rectangle
    \begin{center}
      \input{tikzfigs/drap.tex}
    \end{center} 
  \end{itemize}
\end{frame}

\begin{frame}{Operations on Double-Rectangle Range Space}
  For a double rectangle approximated \emph{I-state} $A(\eta_k) = R_1 \cup R_2$:
  \begin{itemize}
  \item \small{Action Update:\\$T_{drect}\left(A(\eta_k), u_k \right)=DRAP\left(X_{free} \cap [A(\eta_k) \oplus \{ u_k \} \oplus DRAP\left(\Theta(u_k)\right)]\right)$}
  \item \small{Observation Update:\\$O_{drect}\left(A(\eta_k), y_k\right)=AABB\left(H(y_k) \cap R_1\right)\cup AABB\left(H(y_k) \cap R_2\right)$}
  \end{itemize}
  \begin{center}  
    \input{tikzfigs/dbrect.tex}
  \end{center}
\end{frame}

\begin{frame}{Simulation}{Double-Rectangle Approximation of I-state}
  \begin{center}
    \includemovie[toolbar,poster,autoplay]{12cm}{6cm}{videos/clutter_dbrect.mp4}
  \end{center}
\end{frame} 

%%%%%%%%%%%%%%%%%%%%%%%%%%%%%%%%%%%%%%%%%%%%%%%%%%%%%%%%%%%%%%%%%%%%%%%%%%%%

\subsection[Experiments]{Experiments and Conclusions}
\begin{frame}{Experimental Results}{Success Rate}
     \small{Comparison with using the true I-state, we conducted experiments using 3
     environments, and 3 range spaces $\mathcal{R}_{disk}$, $\mathcal{R}_{rect}$,
     and  $\mathcal{R}_{drect}$. \\
     In each environment we collect the success rate of completing the navigation
     task. 
     \begin{enumerate}
     \item The number of landmarks $N$: between $5$ and $250$ in increments of $5$
     \item For each $N$, run the experiment 15 trials with different landmark
       distributions by assigning distinct random seeds
     \end{enumerate}
    }
    \only<1>{
        \centering
        \begin{figure}
          \includegraphics[width=0.65\textwidth]{figs/blank}
        \end{figure}
        Obstacle-free Environment
    }
    \only<2>{
        \centering
        \begin{figure}
          \includegraphics[width=0.67\textwidth]{Data/expnumblank}
        \end{figure}
        Success Rate for the Obstacle-free Environment
    }
    \only<3>{
        \centering
        \begin{figure}
            \includegraphics[width=0.65\textwidth]{figs/clutter}
        \end{figure}
        Obstacle-cluttered Environment
    }
    \only<4>{
        \centering
        \begin{figure}
            \includegraphics[width=0.67\textwidth]{Data/expnumclutter}
        \end{figure}
        Success Rate for the Obstacle-cluttered Environment
    }
    \only<5>{
        \centering
        \begin{figure}
            \includegraphics[width=0.33\textwidth]{figs/office}
        \end{figure}
        Office-like Environment
    }
    \only<6>{
        \centering
        \begin{figure}
            \includegraphics[width=0.67\textwidth]{Data/expnumcse}
        \end{figure}
        Success Rate for the Office-like Environment
    }
\end{frame}

\input{frames/page16.tex}

\input{frames/page17.tex}

\input{frames/page18.tex}

\begin{frame}{Experimental Results}{Success Rate (Cont'd)}
 \small{Comparison with using the true I-state, we conducted experiments using 3
 environments, and 3 range spaces $\mathcal{R}_{disk}$, $\mathcal{R}_{rect}$,
 and  $\mathcal{R}_{drect}$. \\
 In each environment we collect the success rate of completing the navigation
 task. 
 \begin{enumerate}
 \item The number of landmarks $N$: between $5$ and $250$ in increments of 5
 \item For each $N$, run the experiment 15 trials with different landmark
   distributions by assigning distinct random seeds
 \end{enumerate}
}
\begin{figure}
  \centering
  \includegraphics[width=0.33\textwidth]{figs/office}
\end{figure}
\end{frame}

\input{frames/page20.tex}

\begin{frame}{Experimental Results}{Computation Time}
  For each environment, we
  \begin{itemize}
  \item randomly place $N = 300$ landmarks 
  \item run the  experiments $10$ times
  \end{itemize} 
  \begin{block}{}
    \small{We compare the average time (in seconds) required to compute the
      approximated \emph{I-state} compared to the high-fidelity polygonal
      representation of the exact \emph{I-state}.}
  \end{block}
\begin{table}
  \footnotesize\centering
    \begin{tabular}{cccc} 
    \hline
    Range & \multicolumn{3}{c}{Computation Time (s)}  \\
    Space & Obstacle-free & Obstacle-cluttered & Office-like\\
    \hline
    $\mathcal{R}_{disk}$ & 0.163  & 0.162   & 0.292  \\ 
    \hline
    $\mathcal{R}_{rect}$ & 0.396   & 0.441  & 0.415  \\
    \hline
    $\mathcal{R}_{dblrect}$ & 1.021  & 1.122  & 1.491  \\
    \hline
    $\mathcal{I}$ & 10.074  & 10.218  & 26.895  \\
    \hline
    \end{tabular}
    \caption{\scriptsize{Average computation time in three environments.}}
\end{table}
\end{frame}

\begin{frame}{Experimental Results}{Approximation Ratio}
  For each environment, we
  \begin{itemize}
  \item randomly place $N = 300$ landmarks 
  \item run the  experiments $10$ times
  \end{itemize} 
  \begin{block}{}
    \small{We compare the average approximation ratio of using each
      approximation methods to approximate the true \emph{I-state}}
    $$\frac{1}{k} \sum_{i=1}^k \frac{\text{AREA}(\eta_i)}{\text{AREA}(A(\eta_i))}$$
  \end{block}
\begin{table}
  \footnotesize\centering
    \begin{tabular}{cccc} 
    \hline
    Range  & \multicolumn{3}{c}{Approximation  Ratio}  \\
    Space  & Obstacle-free & Obstacle-cluttered & Office-like\\
    \hline
    $\mathcal{R}_{disk}$ & 0.155  & 0.155   & 0.220 \\ 
    \hline
    $\mathcal{R}_{rect}$  & 0.642   & 0.632  & 0.661 \\
    \hline
    $\mathcal{R}_{dblrect}$ & 0.684 & 0.691  & 0.720 \\
    \hline
    $\mathcal{I}$ & 1.000  & 1.000  & 1.000 \\
    \hline
    \end{tabular}
    \caption{\scriptsize{Average approximation ratio in three environments.}}
\end{table}
\end{frame}

\begin{frame}{Conclusions}
  \begin{enumerate}
  \item CGA is effective for representing a robot's uncertain information about
    the current state
  \item The form of double-rectangle is more accurate in approximating the non-convex
    \emph{I-state}
  \item The robot can complete the navigation task using approximated I-state with
    low approximation accuracy
  \end{enumerate}
\end{frame}



\section{Distributed Multi-Robot Formation [ICRA 2014]}
\subsection[Overview]{Overview}
\begin{frame}{Related Work}{Formation using virtual force}
   \begin{columns}[T] 
    \begin{column}{.45\textwidth}
      \scriptsize{W. Spears, D. Spears, J. Hamann and R. Heil, 2004}
      \begin{figure}
        \centering
        \includegraphics[height=1in]{figs/spears1.png}
      \end{figure}
    \end{column}%
    \begin{column}{.45\textwidth}
      \scriptsize{I. Navarro, J. Pugh, A. Martinoli, and
        F. Matia, 2008}
      \begin{figure}
        \centering
        \includegraphics[height=1in]{figs/navarro.png}
      \end{figure}      
    \end{column}
  \end{columns}
  \vspace{3mm}
  \begin{columns}[T] 
    \begin{column}{.45\textwidth}
      \begin{figure}
        \centering
        \includegraphics[height=1in]{figs/spears2.png}     
      \end{figure}  
    \end{column}%
    \begin{column}{.45\textwidth}
      \scriptsize{S. Prabhu, W. Li, J. McLurkin, 2012}
      \begin{figure}
        \centering
        \includegraphics[height=1in]{figs/james.png}
      \end{figure}
    \end{column}
  \end{columns}
\end{frame}

\begin{frame}{Motivation}{}
  \textbf{Question: How to use one algorithm to generate
      various (repeating) lattice pattern formations?}
  \begin{columns}
    \begin{column}{.45\textwidth}
      \begin{figure}
        \centering
        \includegraphics[height=1.5in]{figs/tessellation2}
      \end{figure}
    \end{column}
    \begin{column}{.45\textwidth}
      \begin{figure}
         \centering
         \includegraphics[height=1.5in]{figs/tessellation1}
       \end{figure}
    \end{column}
  \end{columns} 
\end{frame}


\begin{frame}{Robot Model}{}
  \begin{columns}[T] % align columns
    \begin{column}{.55\textwidth}
      \begin{itemize}
      \item Differential drive robots
      \item Each robot has a unique \textbf{ID}
      %\item Use a vector $p = [x, y,
       % \theta]^T$ to represent robot's \textbf{pose}
      \item Each robot has a \textbf{range} within which it can
        sense and communicate with other robots
      \item Each robot gets \textbf{observation} of its neighbors'
        IDs and relative poses in its body frame
      \end{itemize}
    \end{column}%
    % right column
    \begin{column}{.45\textwidth}
        \begin{figure}
    \centering
    \begin{tikzpicture}[scale=0.55]
        \draw[dotted, violet, fill=violet!10] (3, 3) circle (3.2);
        \draw[fill=red] (3,3) -- (2.75,3) -- (3.25,3.25) -- (3,2.75) -- cycle;
        \node[color=red] at (3, 2.3) {$r_i$};           
        \draw[fill=blue!50] (0.5,4.5) -- (0.33,4) -- (0.5,4.125) -- (0.67,4)-- cycle;
        \draw[fill=blue!50] (4,5) -- (3.75,5.5) -- (3.75,5.25) -- (3.5,5.25)-- cycle;
        \draw[fill=blue!50] (1,1.5) -- (1.25,1) -- (1.25,1.25) -- (1.5,1.25)-- cycle;
        \draw[fill=blue!50] (5,2.92) -- (4.5,2.75) -- (5,2.58) -- (4.875,2.75)-- cycle;
        \draw[fill=green!50] (-0.5,0.5) -- (-0.5,0.75) -- (-0.75,0.25) -- (-0.25,0.5) -- cycle;
    \end{tikzpicture}
\end{figure}
\begin{center}
    Robot $r_i$ has 4 neighbors
\end{center}
    \end{column}%
  \end{columns}
\end{frame}


%%%%%%%%%%%%%%%%%%%%%%%%%%%%%%%%%%%%%%%%%%%%%%%%%%%%%%%%
\subsection[Lattice Graph]{Lattice Graph}
% definition of lattice graph

\begin{frame}{Input: Lattice Graph}
  \begin{definition}
    A \textbf{lattice graph} is a strongly connected directed
      multi-graph in which each edge $e$ is labeled with a rigid body
      transformation $T(e)$ and each $\edge{v}{T(e)}{w}$ has an
      inverse edge $\edge{w}{T(e)^{-1}}{v}$.  
    \end{definition}
    \begin{columns}[T] % align columns
      \begin{column}{.45\textwidth}
        \begin{figure}
  \centering
  \begin{tikzpicture}[->,>=stealth',shorten >=5pt,auto,node distance=1cm]
    \tikzstyle{every state}=[fill=blue,draw=none, text=white]
    \node[state, scale=0.6] (A)         {$0$};
    \path (A) edge [loop above] node {\footnotesize{Tr(0, 40)}} (A)
    edge [loop left]  node {\footnotesize{Tr(-40,0)}} (A)
    edge [loop below] node {\footnotesize{Tr(0, -40)}} (A)
    edge [loop right] node {\footnotesize{Tr(40, 0)}} (A);
  \end{tikzpicture}
\end{figure}
      \end{column}%
      \begin{column}{.45\textwidth}
        \begin{figure}
          \centering
          \includegraphics[scale=1]{figs/squarelattice}
        \end{figure}
      \end{column}%
    \end{columns}
\end{frame}

\begin{frame}{Self-Consistent Lattice Graph}
  \begin{definition}
    \small{Given a range $\phi > 0$, call a lattice graph
      \textit{self-consistent} for this range if, for any two paths with the
      same starting node,
      $$\lgpath{v}{T(e_v^{k})}{T(e_m^{w})}{w}, \text{ and }\lgpath{v}{T(e_v^{j})}{T(e_n^{u})}{u},$$
      for which the distance between two ending nodes $w$ and $u$ is
      less than or equal to $\phi$, there exist edges between $w$ and
      $u$ with right transformations.}
  \end{definition}
  \only<1>{
    \begin{columns}[T] % align columns
      \begin{column}{.45\textwidth}
        %\begin{figure}
        \centering
        \begin{tikzpicture}
          \tikzstyle{every state}=[fill=red!50, draw=none]
          \node[state, scale=0.7] (A) at (2.5,0)  {\large{$v$}};
          \node[state, scale=0.7] (C) at (0, 0.75) {\large{$w$}};
          \node[state, scale=0.7] (D) at (0, -0.75) {\large{$u$}};
          \draw[arc] (A) to[out=150, in=0] (C);
          \draw[arc] (A) to[out=-150,in=0] (D);
          \draw[arc] (C) to[out=-30, in=170] (A);
          \draw[arc] (D) to[out=30, in=-170] (A);
          \node[] at (1, -1.2) {\scriptsize{$\left(0, l, 0\right)$}};
          \node[] at (1, 1.2) {\scriptsize{$\left(\dfrac{l}{2}, l, 0\right)$}};
          % \node[] at (-0.5, 0) {\scriptsize{$(\dfrac{\range}{2}, 0, 0)$}};
        \end{tikzpicture}
\end{figure}
        \centering
        \begin{figure}
          \begin{tikzpicture}[->,>=stealth',node distance=2cm]
            \tikzstyle{every state}=[fill=myred, text=white,draw=none]
            \node[state, scale=0.7] (A)    {$0$};
            \path (A) edge [loop right] node {\footnotesize{(1, 0, $45^\circ$)}} (A)
            edge [loop left] node {} (A); %{\footnotesize{(-1,0,$45^\circ$)}} (A);
          \end{tikzpicture}
        \end{figure}
        \small{Self-consistent Lattice Graph}
      \end{column}%
      \begin{column}{.45\textwidth}
        %\begin{figure}
        \centering
        \begin{tikzpicture}
          \tikzstyle{every state}=[fill=red!50, draw=none]
          \node[state, scale=0.7] (A) at (2.5,0)   {\large{$v$}};
          \node[state, scale=0.7] (C) at (0, 0.75) {\large{$w$}};
          \node[state, scale=0.7] (D) at (0, -0.75) {\large{$u$}};
          \draw[arc] (A) to[out=150, in=0] (C);
          \draw[arc] (A) to[out=-150,in=0] (D);
          \draw[arc] (C) to[out=-30, in=170] (A);
          \draw[arc] (D) to[out=30, in=-170] (A);
          \draw[arc] (C) to[out=-60, in=60] (D);
          \draw[arc] (D) to[out=120, in=-120] (C);
          \node[] at (1, -1.2) {\scriptsize{$\left(0, l, 0\right)$}};
          \node[] at (1, 1.2) {\scriptsize{$\left(\dfrac{l}{2}, l, 0\right)$}};
          \node[] at (-1, 0) {\scriptsize{$\left(\dfrac{l}{2}, 0, 0\right)$}};
        \end{tikzpicture}
        % \caption{A lattice graph that is self-consistent when $\phi > l$.}
      \end{figure}
        \begin{figure}
          \includegraphics[width=0.45\columnwidth]{figs/octagon}
        \end{figure}
      \end{column}%
    \end{columns}
  }
  \only<2>{
    \begin{columns}[T] % align columns
      \begin{column}{.45\textwidth}
        %\begin{figure}
        \centering
        \begin{tikzpicture}
          \tikzstyle{every state}=[fill=red!50, draw=none]
          \node[state, scale=0.7] (A) at (2.5,0)  {\large{$v$}};
          \node[state, scale=0.7] (C) at (0, 0.75) {\large{$w$}};
          \node[state, scale=0.7] (D) at (0, -0.75) {\large{$u$}};
          \draw[arc] (A) to[out=150, in=0] (C);
          \draw[arc] (A) to[out=-150,in=0] (D);
          \draw[arc] (C) to[out=-30, in=170] (A);
          \draw[arc] (D) to[out=30, in=-170] (A);
          \node[] at (1, -1.2) {\scriptsize{$\left(0, l, 0\right)$}};
          \node[] at (1, 1.2) {\scriptsize{$\left(\dfrac{l}{2}, l, 0\right)$}};
          % \node[] at (-0.5, 0) {\scriptsize{$(\dfrac{\range}{2}, 0, 0)$}};
        \end{tikzpicture}
\end{figure}
        \centering
        \begin{figure}
          \begin{tikzpicture}[->,>=stealth',node distance=2cm]
            \tikzstyle{every state}=[fill=myred, text=white, draw=none]
            \node[state, scale=0.7] (A)    {$0$};
            \path (A) edge [loop right] node {\footnotesize{(1, 0, $45\sqrt{2}^\circ$)}} (A)
            edge [loop left] node {} (A); %{\footnotesize{(-1,0,$45^\circ$)}} (A);
          \end{tikzpicture}
        \end{figure}
        \small{Non-self-consistent Lattice Graph}
      \end{column}%
      \begin{column}{.45\textwidth}
        %\begin{figure}
        \centering
        \begin{tikzpicture}
          \tikzstyle{every state}=[fill=red!50, draw=none]
          \node[state, scale=0.7] (A) at (2.5,0)   {\large{$v$}};
          \node[state, scale=0.7] (C) at (0, 0.75) {\large{$w$}};
          \node[state, scale=0.7] (D) at (0, -0.75) {\large{$u$}};
          \draw[arc] (A) to[out=150, in=0] (C);
          \draw[arc] (A) to[out=-150,in=0] (D);
          \draw[arc] (C) to[out=-30, in=170] (A);
          \draw[arc] (D) to[out=30, in=-170] (A);
          \draw[arc] (C) to[out=-60, in=60] (D);
          \draw[arc] (D) to[out=120, in=-120] (C);
          \node[] at (1, -1.2) {\scriptsize{$\left(0, l, 0\right)$}};
          \node[] at (1, 1.2) {\scriptsize{$\left(\dfrac{l}{2}, l, 0\right)$}};
          \node[] at (-1, 0) {\scriptsize{$\left(\dfrac{l}{2}, 0, 0\right)$}};
        \end{tikzpicture}
        % \caption{A lattice graph that is self-consistent when $\phi > l$.}
      \end{figure}
        \begin{figure}
          \includegraphics[width=0.45\columnwidth]{figs/badlg-lat}
        \end{figure}
      \end{column}%
    \end{columns}
  }
\end{frame}


\begin{frame}{Role Function}
  \begin{definition}
    \small{Given a lattice graph $G=(V, E)$ and a set of robots $R = \{
    r_1, \ldots, r_n \}$, $R$ \textbf{satisfies} $G$ if
    there exists a role function $f: R \rightarrow V$ that preserves
    the neighborhood structure of $G$.
    \\
    Specifically, for any $i$ and $j$, if $r_i$ and $r_j$ are neighbors, 
    there must exist an edge
    $e_{v}^w: \edge{f(r_i)}{}{f(r_j)}$ in $E$, such that
    $ T(r_j) = T(r_i) T(e_{v}^w)$.}
  \end{definition}
  \input{tikzfigs/satisfy.tex}
\end{frame}


\begin{frame}{Evaluation Criteria}
  \small{\begin{itemize}
  \item \textbf{Efficiency}\\
    \textbf{Execution time:} the smallest time when all the robots reach positions and stay
  \item \textbf{Quality} \\ 
    \textbf{Fulfillment ratio}: for a robot $r_i$ with
    $N_i$ neighbors and $E_i$ outgoing edges
    $$Q = \dfrac{1}{n}\sum\limits_{i=1}^n \frac{N_i}{E_i}$$
  \end{itemize}
  }
  \begin{columns}[T] 
    \begin{column}{.35\textwidth}
        \begin{figure}
    \centering
    \begin{tikzpicture}[scale=0.75, distance=1cm]
          %->,>=stealth',shorten >=5pt,auto,node          ]
          \tikzstyle{every state}=[fill=myred, draw=none, text=white]
          \node[state, scale=0.5] (A) at (0,0)    {$0$};
          \node[state, scale=0.5, fill=mycyan] (B) at (3,0)  {$1$};
          \path (A) edge [bend left=10] node {\scriptsize{Tr(0, 40)}} (B)
                (A) edge [bend left=45] node {\scriptsize{Tr(-35,-20)}} (B)
                (A) edge [bend left=90] node {\scriptsize{Tr(35,-20)}} (B)
                (B) edge [bend left=10] node {\scriptsize{Tr(-40,0)}} (A)
                (B) edge [bend left=45] node {\scriptsize{Tr(35,20)}} (A)
                (B) edge [bend left=90] node {\scriptsize{Tr(-35,20)}} (A);
    \end{tikzpicture}
\end{figure}
    \end{column}%
    \begin{column}{.55\textwidth}
      \begin{figure}
        \centering
        \includegraphics[width=0.75\linewidth]{figs/bad-hexagon}
      \end{figure}
      \begin{figure}
        \centering
        \includegraphics[scale=0.75]{figs/good-hexagon}
      \end{figure}
    \end{column}%
  \end{columns}
\end{frame}


%%%%%%%%%%%%%%%%%%%%%%%%%%%%%%%%%%%%%%%%%%%%%%%%%%%%%%%%
\subsection[Algorithm]{Algorithm}
\begin{frame}{Algorithm}
  \textbf{General Description}
  \begin{columns}[T] % align columns
   \begin{column}{.45\textwidth}
     Robot broadcasts message containing its
     \begin{itemize}
     \item \textcolor{red}{authority} 
     \item \textcolor{red}{matching}
     \end{itemize}

     \begin{enumerate}
     \item Form tree structure
     \item Select roles
     \item Local task assignment
     \item Make movement decision
     \end{enumerate}
    \end{column}%
    \begin{column}{.45\textwidth}
      \input{tikzfigs/algoanimate.tex}
    \end{column}%
  \end{columns}
\end{frame}


\subsubsection[Algorithm: Authority]{Authority Tree}

\begin{frame}{Authority}{Level of Importance}
  \textbf{Define authority and comparison operator}
  \begin{definition}
    \small{An \textbf{authority} is an ordered list of robot IDs
      $$(\id_1, \ldots, \id_k)$$
      The first ID in the list, $\id_1$ is called the \textbf{root} ID.
      The final ID in the list, $\id_k$ is called the \textbf{sender} ID.}
  \end{definition}
  \begin{definition}
    \small{Authority $A_2$ is \textbf{higher than} $A_1$ if:}
    \begin{itemize}
    \item \small{root ID of $A_2 >$ root ID of $A_1$, or}
    \item \small{length of $A_2 <$  length of $A_1$ if they have the same root, or}
    \item \small{sender ID of $A_2 >$ sender ID of $A_1$ if they have the same root and length}
    \end{itemize}
  \end{definition}
\end{frame}

\begin{frame}{1.Construct Authority Tree}{Decide to be root or descendant}
  \textbf{Key Idea:} Each robot forms an authority containing only its
  own ID, compares it with the authorities of remaining messages.
  %The robots use these authorities to establish a collection of authority trees
  \begin{columns}[T] % align columns
    \begin{column}{.45\textwidth}
      \begin{itemize}
      \item {\textcolor{red}{Root}: Robot who has the highest authority}
      \item {\textcolor{blue}{Descendant}: Robot selects the neighbor who sends
          the highest authority and has matching with it as its parent}
      \item {\textcolor{orange}{Orphan}: Robot who has no matching}
      \end{itemize}  
      \footnotesize{[right] Assume each robot has maximum $2$ outgoing
        edges.}
    \end{column}%
    \begin{column}{.5\textwidth}
       \input{tikzfigs/authtree.tex}
    \end{column}
  \end{columns}
\end{frame}

\subsubsection[Algorithm: Task Assignment]{Task Assignment}

\begin{frame}{Matching}{}
  \begin{definition}    
    \small{A \textbf{matching} for a robot is a function $\eta : \{\id(r_a),
      \id(r_b), \ldots \} \rightarrow \{\emptyset,e_{u}^v, e_{u}^w,
      \ldots\} $ that associates each neighbor ID with either a lattice
      graph edge from its role vertex or with the null value
      $\emptyset$.}
  \end{definition}
  \begin{center}
    \input{tikzfigs/matching.tex}
  \end{center}
\end{frame}

\begin{frame}{2. Local Task Assignment}{Hungarian Algorithm}
  To compute an optimal matching of a robot with $N$ neighbors and $E$
  out-going edges of its role in the lattice graph, define a weight
  matrix of size $\min(N,E) \times E$ and apply
  \textcolor{scred}{Hungarian Algorithm} (Harold W. Kuhn, 1955).
    \begin{columns}[T] % align columns
      \begin{column}{.45\textwidth}
        \begin{enumerate}
        \item \small{Each row corresponds to a neighbor}
        \item \small{Each column corresponds to an out-going edge of robot's role}
        \item \small{The entries of the matrix are the Euclidean distance
          between current position of each neighbor and the desired
          position if matched with a lattice graph edge}
        \end{enumerate}
      \end{column}%
      \begin{column}{.45\textwidth}
        \vspace{3mm}
        \input{tikzfigs/taskassign.tex}
        \vspace{3mm}
        \begin{flushleft}
          \footnotesize{5 neighbors, 4 out-going edges.}
        \end{flushleft}
      \end{column}%
    \end{columns}
\end{frame}


\subsubsection[Algorithm: Motion]{Motion Strategy}

\begin{frame}{3. Robot Movement Strategy}{}
  \input{tikzfigs/moving.tex}
\end{frame}

\begin{frame}{Bounded Movement}{}
  \textbf{Goal: Keep descendant stay in the range of its parent}
    \begin{columns}[T] % align columns
      \begin{column}{.5\textwidth}
        \begin{itemize}
        \item \small{Descendant can reach anywhere in set $P$
            (\textcolor{blue}{blue circle}) at next stage}
        \item \small{Descendant is guaranteed to be observed in set $O$
            (\textcolor{red}{Red circle}) by its parent at next stage}
        \item \small{The real destination for descendant is the closest point
            on the boundary of the intersection ($O\cap P$) to the
            assigned destination}
        \end{itemize}
      \end{column}%
      \begin{column}{.4\textwidth}
        \input{tikzfigs/motion.tex}     
      \end{column}%
    \end{columns} 
\end{frame}

%%%%%%%%%%%%%%%%%%%%%%%%%%%%%%%%%%% 
\subsection{Conclusions}

\begin{frame}{Algorithm Description}
  \begin{columns}[T] % align columns
    \begin{column}{.5\textwidth}
      \small{
        Robot broadcasts message containing:
        \begin{itemize}
        \item \textcolor{blue}{HighestID} 
        \item \textcolor{blue}{RootID}: HighestID among all Stable, Vacancy
        \item \textcolor{purple}{Vacancy}
        \item \textcolor{green}{RelocateID}: HighestID among all Unstable 
        \item \textcolor{orange}{Subtree}: is a successor of RelocateID
        \end{itemize}
      }
      %% \begin{enumerate}
      %% \item Tree rooted at HighestID stable robot with Vacancy %(Communication Timeout)
      %% \item Subtree rooted at HighestID unstable robot  
      %% \item Subtree moves with NCLB policy
      %% \item Relocate Robot becomes \textcolor{blue}{stable} after reaching destination
      %% \end{enumerate}
    \end{column}%
    \begin{column}{.4\textwidth}
      \begin{animateinline}[
  begin={%
    \begin{tikzpicture}%
      [post/.style={->,>=stealth', semithick, draw=blue},
        node/.style={circle,fill=red!20,draw,font=\sffamily\small}]%
      \useasboundingbox (0,0) rectangle (4,5.5);
  },
  end={\end{tikzpicture}}
]{10}
  \draw[fill=red] (3,3) -- (2.75,3) -- (3.25,3.25) -- (3,2.75) -- cycle;
  \node[color=blue] at (3, 2.3) {$r_3$};      
  \draw[fill=red] (0.5,4.5) -- (0.33,4) -- (0.5,4.125) -- (0.67,4) -- cycle;
  \node[color=blue] at (0.25, 3.75) {$r_4$};   
  \draw[fill=red] (1,1.5) -- (1.25,1) -- (1.25,1.25) -- (1.5,1.25) -- cycle;
  \node[color=blue] at (1.75, 1.25) {$r_2$};   
  \draw[fill=red] (3,4.75) -- (2.75,4.75) -- (3.25,5) -- (3,4.5) -- cycle;
  \node[color=blue] at (3, 4.25) {$r_1$};   
  \newframe*
  \multiframe{10}{rP = 0.1 +.1}{ 
    \draw[fill=red] (3,3) -- (2.75,3) -- (3.25,3.25) -- (3,2.75)  -- cycle;
    \node[color=black] at (1.75,5.2) {Broadcast Message};
    \node[color=blue] at (3, 2.3) {$r_3$};  
    \draw[fill=red] (0.5,4.5) -- (0.33,4) -- (0.5,4.125) -- (0.67,4) -- cycle;
    \node[color=blue] at (0.25, 3.75) {$r_4$};   
    \draw[fill=red] (1,1.5) -- (1.25,1) -- (1.25,1.25) -- (1.5,1.25) -- cycle;
    \node[color=blue] at (1.75, 1.25) {$r_2$};   
    \draw[fill=red] (3,4.75) -- (2.75,4.75) -- (3.25,5) -- (3,4.5) -- cycle;
    \node[color=blue] at (3, 4.25) {$r_1$};   
    \path (3,3) -- (1.25, 1.25) node[pos=\rP] (q){};
    \draw[post] (3,3) -- (q.west);
    
    \path (3,3) -- (0.5, 4.125) node[pos=\rP] (q){};
    \draw[post] (3,3) -- (q.west);
    
    \path (1.5, 1.25) -- (3,2.5) node[pos=\rP] (q){};
    \draw[post] (1.5, 1.25) -- (q.west);
    
    %\path (1.25, 1.5) -- (0.5, 4.125) node[pos=\rP] (q){};
    %\draw[post] (1.25, 1.5) -- (q.south);
    
    \path (0.5, 4.5) -- (3.25, 3.25) node[pos=\rP] (q){};
    \draw[post] (0.5, 4.5) -- (q.north);

    %\path (0.33, 4) -- (1, 1.5) node[pos=\rP] (q){};
    %\draw[post] (0.33, 4) -- (q.north);

    \path (3,3) -- (3.25,4.75) node[pos=\rP] (q){};
    \draw[post] (3,3) -- (q.south);

    \path (3,4.75) -- (2.75,3) node[pos=\rP] (q){};
    \draw[post] (3,4.75) -- (q.north);

    \path (0.5,4.5) -- (3,4.75) node[pos=\rP] (q){};
    \draw[post] (0.5,4.5) -- (q.west);

    \path (3,4.6) -- (0.6,4.35) node [pos=\rP] (q){};
    \draw[post] (3,4.6) -- (q.east);
  }
  \newframe*
  \multiframe{10}{rP = 0.1 +.1}{
    \node[color=black] at (1.75, 5.2) {Form Tree};
    \draw[fill=green] (3,3) -- (2.75,3) -- (3.25,3.25) -- (3,2.75)  -- cycle;
    \node[color=blue] at (3.2, 2.3) {$r_3$};
    \node[color=blue] at (3.2, 2) {relocate};
    \draw[fill=blue] (0.5,4.5) -- (0.33,4) -- (0.5,4.125) -- (0.67,4) -- cycle;
    \node[color=blue] at (0.25, 3.75) {$r_4$};
    \node[color=blue] at (0.5, 3.35) {root};
    \draw[fill=orange] (1,1.5) -- (1.25,1) -- (1.25,1.25) -- (1.5,1.25) -- cycle;
    \node[color=blue] at (1.75,1.25) {$r_2$};
    \draw[fill=red] (3,4.75) -- (2.75,4.75) -- (3.25,5) -- (3,4.5) -- cycle;
    \node[color=blue] at (3, 4.25) {$r_1$};  
    %\path (3,3) -- (1.25, 1.25) node[pos=\rP] (q){};
    \draw[post] (3,3) -- (1.25,1.25); % r3 --> r2 
    \draw[post] (0.5, 4.25) -- (3,4.6); % r4 --> r1
    % \path (3,3) -- (0.5, 4.125) node[pos=\rP] (q){};
    \draw[post] (0.5, 4.25) -- (3,3); % r4 --> r3
    \draw[color=purple] (1,3) -- (0.75,3) -- (1.25,3.25) -- (1,2.75) -- cycle;
    \node[color=purple] at (1,2.5) {vacancy};
    \draw[dashed](0.5,4.125) -- (1,3);
  }
  \newframe*
  \multiframe{10}{rP = 0.1 +.1}{ 
    \node[color=black] at (1.75, 5.2) {Movement};
    %    \draw[fill=green] (3,3) -- (2.75,3) -- (3.25,3.25) -- (3,2.75)  -- cycle;
    %\node[color=blue] at (3.2, 2.3) {$r_3$};
    %\node[color=blue] at (3.2, 2) {relocate};
    \draw[fill=blue] (0.5,4.5) -- (0.33,4) -- (0.5,4.125) -- (0.67,4) -- cycle;
    \node[color=blue] at (0.25, 3.75) {$r_4$};
    \draw[fill=orange] (1,1.5) -- (1.25,1) -- (1.25,1.25) -- (1.5,1.25) -- cycle;
    \node[color=blue] at (1.75,1.25) {$r_2$};
    \draw[fill=red] (3,4.75) -- (2.75,4.75) -- (3.25,5) -- (3,4.5) -- cycle;
    \node[color=blue] at (3, 4.25) {$r_1$};  
    %\path (3,3) -- (1.25, 1.25) node[pos=\rP] (q){};
    \draw[post] (1,3) -- (1.25,1.25); % r3 --> r2 
    \draw[post] (0.5, 4.25) -- (3,4.6); % r4 --> r1
    % \path (3,3) -- (0.5, 4.125) node[pos=\rP] (q){};
    \draw[post] (0.5, 4.25) -- (1,3); % r4 --> r3
    \draw[fill=blue] (1,3) -- (0.75,3) -- (1.25,3.25) -- (1,2.75) -- cycle;
    %\node[color=purple] at (1,2.5) {vacancy};
    \node[color=blue] at (1.2, 2.7) {$r_3$};
  }
  \newframe*
  \multiframe{10}{rP = 0.1 +.1}{ 
    \node[color=black] at (1.75, 5.2) {Next Round ...};
    \draw[fill=blue] (0.5,4.5) -- (0.33,4) -- (0.5,4.125) -- (0.67,4) -- cycle;
    \node[color=blue] at (0.25, 3.75) {$r_4$};
    \draw[fill=green] (1,1.5) -- (1.25,1) -- (1.25,1.25) -- (1.5,1.25) -- cycle;
    \node[color=blue] at (1.75,1.25) {$r_2$ (relocate)};
    \draw[fill=red] (3,4.75) -- (2.75,4.75) -- (3.25,5) -- (3,4.5) -- cycle;
    \node[color=blue] at (3, 4.25) {$r_1$};  
    %\path (3,3) -- (1.25, 1.25) node[pos=\rP] (q){};
    \draw[post] (1,3) -- (1.25,1.25); % r3 --> r2 
    \draw[post] (0.5, 4.25) -- (3,4.6); % r4 --> r1
    % \path (3,3) -- (0.5, 4.125) node[pos=\rP] (q){};
    \draw[post] (0.5, 4.25) -- (1,3); % r4 --> r3
    \draw[fill=blue] (1,3) -- (0.75,3) -- (1.25,3.25) -- (1,2.75) -- cycle;
    %\node[color=purple] at (1,2.5) {vacancy};
    \node[color=blue] at (1.2, 2.7) {$r_3$ (root)};
    \draw[dashed] (1,3) -- (2.5,2.625);
    \draw[color=purple] (2.5,3) -- (2.33,2.5) -- (2.5,2.625) -- (2.67,2.5) -- cycle;
    \node[color=purple] at (2.5, 3.2) {vacancy};
  }
\end{animateinline}
    \end{column}%
  \end{columns}
\end{frame}

%%%%%%%%%%%%%%%%
\end{document}
